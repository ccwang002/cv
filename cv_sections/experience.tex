% !TeX root = ../cv.tex
\section{Experience}

\begin{entrylist}

\item \textbf{Mandatory National Service} \hfill Taipei, Taiwan\\
Substitute Services Draftee \hfill
Mar 2022--Mar 2023
\begin{detaillist}
    % \item Assisted special education students at a public high school by providing daily care and support
    \item Optimized the user experience of Taiwan's national long-term care system by contributing to a \href{https://d4sg.org/}{Data for Social Good} project in partnership with Taiwan's Ministry of Health and Welfare
    \item Improved the call center operations of the national long-term care system by designing a data-driven staffing system and updating standard operating protocols by analyzing call transcriptions and service applications
\end{detaillist}

\item \textbf{Washington University in St. Louis} \hfill St. Louis, USA\\
PhD Research, \href{https://dinglab.wustl.edu/}{Li Ding Lab} \hfill
Sep 2016--Dec 2021
\begin{detaillist}
    % \item Thesis: Building a toolbox and insights toward proteogenomic characterization of glioblastoma
    \item Collaborated with leading cancer research communities including The Cancer Genome Atlas (TCGA), Clinical Proteogenomic Tumor Analysis Consortium (CPTAC), and Human Tumor Atlas Network (HTAN)
    \item Led the data analysis team of CPTAC glioblastoma study, coordinating efforts across multiple institution, to identify different patient stratifications and druggable targets in signaling pathways using multi-omics analysis with multiplexed mass spectrometry and bulk genomics
    \item Investigated tumor heterogeneity and microenvironment using single cell sequencing, spatial transcriptomics and multiplexed imaging (Hyperion, CODEX)
    \item Developed tools, processing pipelines, and databases (\href{https://ptmcosmos.wustl.edu/}{PTMcosmos}, \href{https://github.com/ding-lab/CharGer}{CharGer}) to analyze multi-modal datasets
    \item Conducted quality control analysis on different versions of the mutation calling pipeline on the \href{https://gdc.cancer.gov/}{NCI Genomics Data Commons}, a platform hosting and reprocessing data of multiple cancer genome programs
\end{detaillist}

\item \textbf{Open Source Software} \hfill Online\\
Contributor \hfill 2015--Present
\begin{detaillist}
    \item Contributed to the development of open source tools and projects including conda-forge, bioconda, and cyvcf2
    \item Demonstrated proficiency in the software development workflow, including testing, packaging, CI/CD, issue triaging, and documentation
\end{detaillist}

\item \href{http://pinkoi.com}{\textbf{Pinkoi}} \hfill Taipei, Taiwan\\
Internship on search improvement \hfill
Summer 2015
\begin{detaillist}
    \item Enhanced the search functionality on \href{http://pinkoi.com}{Pinkoi}, Asia's leading online marketplace for original design goods
    \item Investigated various Mandarin/Chinese text segmentation methods in ElasticSearch
    \item Developed a product recommendation system when user's search query failed to match any exact product by implementing a Word2Vec-based semantic search
\end{detaillist}

\item \textbf{Microsoft Research Asia} \hfill Beijing, China\\
Research Internship, \href{https://www.microsoft.com/en-us/research/people/echang/}{Eric Chang Group} \hfill
2014--2015
\begin{detaillist}
    \item Investigated the use of deep learning for histopathology image and pathway analysis in the brain and colon cancer
    \item Achieved highly accurate histopathology image classification and segmentation using convolutional neural network (CNN) based models
    \item Analyzed the CNN activation features and discovered that the model was able to
    accurately identify key morphological features and ignore irrelevant background noise as identified by pathologists
\end{detaillist}

\item \textbf{National Taiwan University} \hfill Taipei, Taiwan\\
Master's Research, Eric Y. Chuang Lab \hfill
2014--2016
\begin{detaillist}
    % \item Thesis: \href{https://doi.org/10.6342/NTU201601295}{BioCloud: an online sequencing analysis platform}
    \item Developed a cloud platform to process and analyze sequencing data, including a web portal, a data processing job queue scheduler, and a database to track data and visualize results
\end{detaillist}

% % Undergraduate research
% \item \textbf{National Taiwan University} \hfill Taipei, Taiwan\\
% Undergraduate Research on genomic data analysis, Eric Y. Chuang Lab \hfill
% 2012--2014\\
% \begin{detaillist}
%	\item Integrative analysis of copy number variation, gene expression, and miRNA regulation on cancer
%	\detail
%	\item Genomic data processing pipeline development for DNA-seq, RNA-seq, and miRNA-seq
% \end{detaillist}
% Undergraduate Research on Bio-MEMS sensor design, Wei-Cheng Tian Lab \hfill
% 2011--2012
% \begin{detaillist}
% 	\detail
% 	\item CMOS-based tactile sensor system for blood pressure measurement
% \end{detaillist}

\end{entrylist}
